% !TEX root = main.tex

\chapter{Lenguaje de programaci\'on}
\label{chapter.lang}

Este cap\'itulo presenta una propuesta de un lenguaje de programaci\'on basado en el modelo \SCCP definido en el Cap\'itulo~\ref{chapter.sccp}. La principal raz\'on para crear este lenguaje es permitir a los programadores tener una medio m\'as sencillo para modelar sistemas distribuidos de informaci\'on. 

La Secci\'on~\ref{ebnf.lang} presenta la sintaxis del lenguaje de programaci\'on en la notaci\'on EBNF. La Secci\'on~\ref{aa.lang} muestra c\'omo el lenguaje esta relacionado con el modelo \SCCP descrito en el Cap\'itulo~\ref{chapter.sccp}. La Secci\'on~\ref{ss.lang} presenta una explicaci\'on de los nuevos elementos del lenguaje. Por \'ultimo, la Secci\'on~\ref{example.lang} contiene algunos ejemplos del funcionamiento del lenguaje de programaci\'on.

%% Secci\'on:  %%
\section{EBNF del lenguaje}
\label{ebnf.lang}

EBNF (del ing\'es, Extended Backus-Naur Form) es un metalenguaje utilizado para definir formalmente la sintaxis de gram\'aticas libres de contexto, que utiliza expresiones regulares para permitir escribir especificaciones compactas. 

Una descripci\'on EBNF es una lista no ordenada de reglas EBNF. Cada una de las reglas EBNF esta compuesta de tres partes: el lado izquierdo, el lado derecho y el s\'imbolo `$::=$' separando los dos lados, este s\'imbolo debe ser leido como ``se define como''~\cite{ebnfdoc}. El lado izquierdo contiene una palabra escrita en min\'uscula, la cual indica el nombre de la regla EBNF. En el lado derecho se encuentra la definici\'on asociada al este nombre. 

Los nombres asignados a las reglas EBNF en el lado izquierdo pueden ser utilizados como parte de la definici\'on asociada a las reglas EBNF, inclusive en la misma regla, y siempre se encuentran delimitados por los caracteres `$\langle$' y `$\rangle$'. 

Las reglas EBNF pueden incluir diez caracteres con significado especial: `$::=$', `$|$', `$+$', `$*$', `$[$', `$]$', `$($', `$)$', `$?$' y `$;$'. Los dem\'as caracteres ademas de los nombres y los listados anteriormente se definen por si mismos (v.gr. letras, d\'igitos, signos de puntuaci\'on). A continuaci\'on se presenta una explicaci\'on de algunos de los caracteres especiales:

\begin{itemize}
\item Cada regla EBNF debe finalizar con el car\'acter `$;$'. 
\item `$*$' es un operador unario posfijo, el cual indica que la expresi\'on aparece cero o m\'as veces. 
\item `$+$' es un operador unario posfijo, el cual indica que la expresi\'on aparece una o m\'as veces.
\item `$?$' es un operador unario posfijo, el cual indica que la expresi\'on aparece cero o una vez.
\item Multiples opciones en la definici\'on de un regla EBNF deben estar separadas por el car\'acter `$|$'.
\item Todas las expresiones delimitadas por comillas simples aparecen de la misma forma en el lenguaje de programaci\'on.
\end{itemize}

\begin{figure}

\begin{flalign*}
\nterm{system} & ::= \nterm{variables}? \nterm{body} \ ; \\
\nterm{variables} & ::= \textterm{var} \nterm{ID}+ \ (\textterm{Int} | \textterm{Bool}) \ ; \\
\nterm{body} & ::= \textterm{begin} \nterm{processLine}+ \ \textterm{end} \ ; \\
\nterm{processLine} & ::= \nterm{process} \textterm{.}  \ ; \\
\nterm{process} & ::= \textterm{tell(} \ \nterm{constraint}  \textterm{)} \ ; \\
	& \bor \textterm{ask} \ (\textterm{<} \ \nterm{location} \textterm{>})? \ \nterm{constraint} \textterm{->}  \nterm{process} \ ; \\
	& \bor \nterm{process} \textterm{||}  \nterm{process} \ ; \\
	& \bor  \textterm{[}  \nterm{process}  \textterm{]\_} \ \nterm{integer} \ ; \\
\nterm{constraint} & ::= \nterm{boolean} \ ; \\
	& \bor \nterm{ID} \ ; \\
	& \bor \nterm{expression} \ ; \\
	& \bor \nterm{constraint} \ \textterm{and} \ \nterm{constraint} \ ; \\
\nterm{location} & ::= \nterm{integer} (\textterm{.} \nterm{integer})* \ ; \\
\nterm{expression} & ::= \nterm{ID} \nterm{operator} ( \nterm{id} \ | \ \nterm{integer} ) \ ; \\
\nterm{operator} & ::= \textterm{>} \ | \ \textterm{<} \ | \ \textterm{=} \ | \ \textterm{=/=} \ | \ \textterm{>=} \ | \ \textterm{<=} \ ; \\
\nterm{boolean} & ::= \textterm{true} \ | \ \textterm{false} \ ; \\
\nterm{integer} & ::= \textnormal{[0-9]}+ \ ; \\
\nterm{id} & ::= \textnormal{[A-Z] [A-Z0-9]}* \ ;  
\end{flalign*}

\caption{EBNF del lenguaje de programaci\'on.}
\label{fig:ebnf}
\end{figure}

En la Figura~\ref{fig:ebnf} se presenta la descripci\'on EBNF del lenguaje de programaci\'on propuesto. La explicaci\'on de cada una de las reglas EBNF se presenta en las secciones~\ref{aa.lang} y ~\ref{ss.lang}.

%% Secci\'on:  %%
\section{aa}
\label{aa.lang}

explain how a subset of the language relates to the \SCCP definition of previous section

\begin{itemize}
\item \cde{system} esta definido
\end{itemize}


%% Secci\'on:  %%
\section{ss}
\label{ss.lang}

explain the new constructs of the language

\begin{itemize}
\item \cde{system} esta definido
\end{itemize}


%% Secci\'on:  %%
\section{Ejemplo}
\label{example.lang}

encode the example from the previous section in the language
encode another example in the language
