% !TEX root = main.tex

\chapter{Lenguaje de programaci\'on}
\label{chapter.lang}

Este cap\'itulo presenta una propuesta de un lenguaje de programaci\'on basado en el modelo \SCCP definido en el Cap\'itulo~\ref{chapter.sccp}. 
%una especificaci�n formal del modelo SCCPen la sintaxis de Maude [4]. La especificaci�n formal consta de un m�dulo funcional que define el identificador de un agente; un m�dulo funcional que define la sintaxis y los tipos requeridos para especificar los dife- rentes tipos de comandos u operadores disponibles en el modelo; y un m�dulo funcional que define la sintaxis y los tipos requeridos para es- pecificar los estados del modelo. Este m�dulo incluye, entre otros, la definici�n de un agente y un proceso. La transferencia de informaci�n en SCCPesta formalizada a partir de de reglas de reescritura que son parte de un m�dulo de sistema.

La Secci\'on~\ref{ebnf.lang} presenta la sintaxis del lenguaje de programaci\'on en la notaci\'on eBNF. La Secci\'on~\ref{.lang} muestra c\'omo el lenguaje esta relacionado con el modelo \SCCP descrito en el Cap\'itulo~\ref{chapter.sccp}. La Secci\'on~\ref{.lang} presenta una explicaci\'on de los nuevos elementos del lenguaje. Por \'ultimo, la Secci\'on~\ref{example.lang} contiene algunos ejemplos del funcionamiento del lenguaje de programaci\'on.

%% Secci\'on:  %%
\section{eBNF del lenguaje}
\label{ebnf.lang}

EBNF (del ing\'es, Extended Backus-Naur Form) es un metalenguaje utilizado para definir formalmente la sintaxis de gram\'aticas libres de contexto, que utiliza expresiones regulares para permitir escribir especificaciones compactas. 

Una descripci\'on EBNF es una lista no ordenada de reglas EBNF. Cada una de las reglas EBNF esta compuesta de tres partes: el lado izquierdo, el lado derecho y el s\'imbolo `$::=$' separando los dos lados, este s\'imbolo debe ser leido como ``se define como''~\cite{ebnfdoc}. El lado izquierdo contiene una palabra escrita en min\'uscula, la cual indica el nombre de la regla EBNF. En el lado derecho se encuentra la definici\'on asociada al este nombre. 

Los nombres asignados a las reglas EBNF en el lado izquierdo pueden ser utilizados como parte de la definici\'on asociada a las reglas EBNF, inclusive en la misma regla, y siempre se encuentran delimitados por los caracteres `$\langle$' y `$\rangle$'. 

Las reglas EBNF pueden incluir nueve caracteres con significado especial: $::=$, $|$, $+$, $*$, $[$, $]$, $($, $)$ y $;$. Los dem\'as caracteres ademas de los nombres y los listados anteriormente se definen por si mismos (v.gr. letras, d\'igitos, signos de puntuaci\'on). Cada regla EBNF debe finalizar con el car\'acter `$;$'.


\begin{flalign*}
\nterm{system} & ::= \nterm{variables}? \nterm{body} \ ; \\
\nterm{variables} & ::= `var' \nterm{ID}+ (`Int' | `Bool') \ ; \\
\nterm{body} & ::= `begin' \nterm{processLine}+ `end' \ ; \\
\nterm{processLine} & ::= \nterm{process} `.'  \ ; \\
\nterm{process} & ::= `tell(' \ \nterm{constraint}  `)' \ ; \\
	& \bor `ask' \ (`<' \ \nterm{location} `>')? \nterm{constraint} `->'  \nterm{process} \ ; \\
	& \bor \nterm{process} \ `||' \ \nterm{process} \ ; \\
	& \bor  `[' \ \nterm{process} \ `]\_' \ \nterm{integer} \ ; \\
\nterm{constraint} & ::= \nterm{boolean} \ ; \\
	& \bor \nterm{ID} \ ; \\
	& \bor \nterm{expression} \ ; \\
	& \bor \nterm{constraint} \ `and' \ \nterm{constraint} \ ; \\
\nterm{location} & ::= \nterm{integer} (`.' \nterm{integer})* \ ; \\
\nterm{expression} & ::= \nterm{ID} \nterm{operator} ( \nterm{ID} \ | \ \nterm{integer} ) \ ; \\
\nterm{operator} & ::= `>' \ | \ `<' \ | \ `=' \ | \ `=/=' \ | \ `>=' \ | \ `<=' \ ; \\
\nterm{boolean} & ::= `true' \ | \ `false' \ ; \\
\nterm{integer} & ::= [0-9]+ \ ; \\
\nterm{ID} & ::= [A-Z] [A-Z0-9]* \ ; 
\end{flalign*}

%% Secci\'on:  %%
\section{aa}
\label{aa.lang}

explain how a subset of the language relates to the \SCCP definition of previous section

%% Secci\'on:  %%
\section{ss}
\label{ss.lang}

explain the new constructs of the language

%% Secci\'on:  %%
\section{Ejemplo}
\label{example.lang}

encode the example from the previous section in the language
encode another example in the language
