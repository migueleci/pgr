\chapter{Preliminares}
\label{p.prelim}

%Secci\'on: L\'ogica de reescritura
\section{L\'ogica de reescritura}
\label{lr.prelim}

L\'ogica de reescritura~\cite{Meseguer199273} es una l\'ogica de cambios
concurrentes. Las reglas de la l\'ogica de reescritura son patrones
generales para acciones b\'asicas que pueden ocurrir concurrentemente con
otras acciones en un sistema concurrente. Por lo tanto, la l\'ogica de
reescritura permite razonar sobre cambios complejos en un sistema,
teniendo en cuenta que los cambios corresponden a las acciones b\'asicas
axiomatizadas por las reglas de reescritura.


 %Subsecci\'on:
 \subsection{Teor\'ia ecuacional}
 \label{te.prelim}

 Una {\em signatura} ordanada por tipos $\Sigma$ es una tupla
 $\Sigma = (S, \leq, F)$ con un conjunto parcialmente ordenado de tipos
 $(S, \leq)$ y un conjunto de s\'imbolos de funci\'on $F$. La relaci\'on
 binaria $\equiv_\leq$ denota la relaci\'on de equivalencia generada por
 $\leq$ sobre $S$ y su extensi\'on a cadenas en $S^*$.

 La colecci\'on de variables $X$ es una familia $S$-indexada
 $X = \{ X_s \}_{s \in S}$ de conjuntos de variables disyuntos
 con cada $X_s$ infinito contable. $T_\Sigma(X)_s$ es el
 {\em conjunto de t\'erminos de tipo s} y el {\em conjunto de t\'erminos simples de tipo s}
 se denota por $T_{\Sigma, s}$. Las expresiones $\tcal_\Sigma(X)$
 y $\tcal_\Sigma$ denotan las correspondientes algebras de
 $\Sigma$-t\'erminos ordenadas por tipos.

 Una {\em $\Sigma$-ecuaci\'on} es una pareja $t = u$ con
 $t \in T_\Sigma(X)_{s_t}$, $u \in T_\Sigma(X)_{s_u}$ y
 $s_t \equiv_\leq s_u$. Una {\em $\Sigma$-ecuaci\'on condicional} es una
 ecuaci\'on condicional $t = u$ \tbf{if} $\gamma$ con $t = u$ una
 $\Sigma$-ecuaci\'on y $\gamma$ una conjunci\'on finita de $\Sigma$-ecuaciones.
 Un {\em tipo al tope} en $\Sigma$ es un tipo $s \in S$ tal
 que si $s' \in S$ y $s \equiv_\leq s'$, entonces $s' \leq s$.

 Una {\em teor\'ia ecuacional} es un par $(\Sigma, E)$, donde $\Sigma$ es
 una signatura y $E$ es una colecci\'on finita de ecuaciones, posiblemente
 condicionales.
 Una teor\'ia ecuacional $\ecal=(\Sigma, E)$ induce la relaci\'on de
 congruencia $=_\ecal$ sobre $T_\Sigma(X)$ definida por $t =_\ecal u$, 
 con $t, u \in T_\Sigma(X)$, s\'i y s\'olo s\'i $\ecal \vdash t=u$ por las
 reglas de deducci\'on para l\'ogica ecuacional ordenada por tipos
 en~\cite{meseguer97}, s\'i y s\'olo s\'i, en~\cite{meseguer97} $t=u$ es v\'alido
 en todos los modelos de $\ecal$.

 Las expresiones $\tcal_{\Sigma/E}(X)$ y $\tcal_{\Sigma/E}$ corresponden
 a las algebras de cocientes inducidas por $=_\ecal$ sobre las algebras de
 t\'erminos $\tcal_\Sigma(X)$ y $\tcal_\Sigma$.
 El algebra $\tcal_{\Sigma/E}$ es llamado el {\em algebra inicial} de
 $(\Sigma, E)$.
 
 Las $\Sigma$-ecuaciones est\'an divididas en un conjunto $A$ de axiomas
 estructurales (tales como asociatividad, conmutatividad y/o identidad)
 y el conjunto $E$ de ecuaciones.

 %Subsecci\'on:
 \subsection{Teor\'ia de reescritura}
 \label{tr.prelim}
 
 Una {\em teor\'ia de reescritura} es una tupla $\rcal = (\Sigma, E, R)$
 con una teor\'ia ecuacional $\ecal_\rcal = (\Sigma, E)$ y un conjunto
 finito de $\Sigma$-reglas $R$. Una {\em teor\'ia de reescritura al tope}
 es una teor\'ia de reescritura $\rcal = (\Sigma, E, R)$, tal que cada regla
 $t \rightarrow u$ \tbf{if} $\gamma \in R$ es tal que $l, r \in T_{\Sigma}(X)_s$
 para alg\'un $s = \cuad{s}$ en $\Sigma$, $l \notin X$, y no hay operadores
 en $\Sigma$ que tengan al tipo $s$ como argumento.

 Una teor\'ia de reescritura $\rcal=(\Sigma, E, R)$ induce la relaci\'on de
 reescritura $\f{\rcal}$ sobre $T_\Sigma(X)$ definida por $t\f{\rcal}u$, 
 con $t, u \in T_\Sigma(X)$, s\'i y s\'olo s\'i una demostraci\'on de reescritura
 de un paso de $\rcal \vdash t \rightarrow u$ puede ser obtenida por las
 reglas de deducci\'on para teor\'ias de reescritura ordenadas por tipos
 en~\cite{bruni06}, s\'i s\'olo s\'i, en~\cite{bruni06} $t \rightarrow u$ es
 v\'alido en todos los modelos de $\rcal$.
 La expresi\'on $\tcal_\rcal = (\tcal_{\Sigma/E},\fa{\rcal}{*})$ denota el
 modelo de alcanzabilidad inicial de $\rcal = (\Sigma, E, R)$~\cite{bruni06}, donde $\fa{\rcal}{*}$ representa la clausura
transitiva-reflexiva de $\f{\rcal}$.

%Secci\'on: Maude
\section{Maude}
\label{m.prelim}

Maude~\cite{maude-book} es un lenguaje declarativo. Un programa en Maude
es una teor\'ia l\'ogica y un c\'alculo en Maude es una deducci\'on l\'ogica que
utiliza los axiomas especificados en la teor\'ia o el programa, seg\'un
corresponde, bajo el sistema deductivo de la l\'ogica de reescritura~\cite{Meseguer199273}.

Maude cuenta con dos tipos de m\'odulos: funcional y de sistema.

 %Subsecci\'on:
 \subsection{M\'odulos funcionales}
 \label{mf.prelim}

 Los m\'odulos funcionales corresponden a teor\'ias ecuacionales y definen
 tipos de datos y operaciones sobre estos tipos de datos.

 Los m\'odulos funcionales de Maude suponen la propiedad de que la
 ecuaciones son consideradas reglas de simplificaci\'on que se usan
 \'unicamanete de izquierda a derecha y su repetida aplicaci\'on reduce un
 t\'ermino a su forma can\'onica, la cual no depende del orden en el que se
 usen las ecuaciones. Este tipo de simplificaci\'on can\'onica es posible si
 la teor\'ia ecuacional asociada a un m\'odulo funcional es
 Church-Rosser~\cite{DBLP:journals/jlp/DuranM12} y terminante~\cite{Lucas2009207}.

 Los m\'odulos funcionales pueden contener: ecuaciones con o sin atributos,
 pertenencias y operadores con sus respectivos atributos. Las ecuaciones
 y pertenencias pueden ser condicionales o incondicionales. Un m\'odulo
 funcional se define con la palabra clave \cde{fmod}. Para m\'as informaci\'on
 sobre conceptos b\'asicos y sintaxis de Maude para m\'odulos funcionales se
 refiere al lector a~\cite{maude-book}.

 %Subsecci\'on:
 \subsection{M\'odulos de sistema}
 \label{ms.prelim}

 Un m\'odulo de sistema especifica una teor\'ia de reescritura. Una teor\'ia de
 reescritura contiene tipos de datos, clases de equivalencia de
 tipos de datos, operadores, y ecuaciones, pertenencias y reglas de
 reescritura, las cuales pueden ser condicionales. De lo anterior se
 puede afirmar que toda teor\'ia de reescritura tiene una teor\'ia ecuacional
 subyacente. Un m\'odulo de sistema se declara con la palabra clave
 \cde{mod}. Para m\'as informaci\'on sobre conceptos b\'asicos y sintaxis de
 Maude para m\'odulos de sistema se refiere al lector a~\cite{maude-book}.

 El conjunto de $\Sigma$-reglas $R$ es coherente con respecto a las
 ecuaciones $E$ m\'odulo $A$, si Maude puede ejecutar un m\'odulo de sistema
 admisible~\cite{maude-book} usando la estrategia de primero simplificar
 un t\'ermino $t$ a su forma $E/A$-can\'onica y luego usar una regla con $R$
 m\'odulo $A$ para lograr el efecto de reescribir con $R$ m\'odulo $E \cup A$.

%Secci\'on: LTL model-checking
\section{{\em LTL model-checking}}
\label{ltl.prelim}

Con {\em LTL model-checking} es posible demostrar propiedades de l\'ogica
lineal temporal para m\'odulos de sistema que tienen una cantidad finita de
estados alcanzables desde un estado unicial dado.

Una estructura de Kripke~\cite{clarke-book} es un sistema (total) de
transiciones sin etiquetar al cual se le agrega una colecci\'on de
predicados de estados unarios en su conjunto de estados.
Suponiendo que dado un m\'odulo de sistema \cde{M}, el cual especifica una
teor\'ia de reescritura $\rcal = (\Sigma,E,R)$, se tiene:

\begin{itemize}
   \item elegido un tipo $k = \cuad{s}$ en \cde{M} para los estados.
   
   \item definido alg\'un conjunto de predicados de estado $\Pi$ y su
	sem\'antica en el conjunto de ecuaciones $D$.
\end{itemize}

Se define la estructura de Kripke $\kcal(\rcal,k)_{\Pi}$ sobre el conjunto
de predicados at\'omicos $AP_{\Pi}$. Dado un estado inicial $\rel{t}{k}$ y
una formula LTL $\varphi \in LTL(AP)_\Pi$ se quiere obtener un
procedimiento para decidir la relaci\'on

$\kcal(\rcal,k)_{\Pi}, \cuad{t} \models \varphi$,

\noindent que significa que desde el estado inicial $\cuad{t}$ se
satisface $\varphi$ en la estructura de Kripke $\kcal(\rcal,k)_{\Pi}$.
En general, esta relaci\'on es indecidible pero puede convertirse en
decidible si se cumplen las siguientes condiciones:

\begin{enumerate}
   \item el conjunto de estados en $\rel{t}{k}$ que son alcanzables desde
	$\cuad{t}$ es finito, y

   \item la teor\'ia de reescritura $\rcal = (\Sigma,E,R)$ especificada
	por el m\'odulo \cde{M} y las ecuaciones $D$, satisfacen que:
	\begin{itemize}
	   \item $E$ y $E \cup D$ son (simples) Church-Rosser~\cite{DBLP:journals/jlp/DuranM12}
		y terminante~\cite{Lucas2009207}, m\'odulo algunos axiomas
		de $A$, con $(\Sigma,E) \subseteq (\Sigma \cup \Pi, E \cup D)$
		una extensi\'on protegida, y

	   \item $R$ es (simple) coherente con relaci\'on a $E$, de nuevo,
		m\'odulo algunos axiomas de $A$.
	\end{itemize}
\end{enumerate}

Bajo estas suposiciones, los predicados de estados $\Pi$ y la relaci\'on de 
transici\'on $\fa{\rcal}{1}$, son calculables y, dada la suposici\'on de
alcanzabilidad, se puede resolver el problema de satisfacci\'on usando un
procedimiento de {\em model-checking}~\cite{clarke-book}.

%Secci\'on: Ret\'iculo Algebraico Completo
\section{{\em Ret\'iculo Algebraico Completo}}
\label{rac.prelim}

Una relaci\'on binaria $\leq$ definida en un conjunto $A$ es un \textit{orden parcial} del 
conjunto $A$ si se mantienen las condiciones \textit{C1}, \textit{C2} y \textit{C3}.

\begin{enumerate}  
	\item [\it{C1}] \ Reflexividad. \ $a \leq a$
	\item [\it{C2}] \ Antisimetr\'ia. \ $a \leq b$ y $b \leq a$ implica $a = b$
	\item [\it{C3}] \ Transitividad. \ $a \leq b$ y $b \leq c$ implica $a \leq b$
\end{enumerate}

Un conjunto no vac\'io con orden parcial es llamado un \textit{conjunto parcialmente 
ordenado}, o m\'as brevemente un \textit{poset} (del ingl\'es, \textit{partially ordered set}).

Un conjunto parcialmente ordenado $L$ es un \textit{ret\'iculo} si y solo si para cada $a$, 
$b$ en $L$ existen sup$\{a,b\}$ e inf$\{a,b\}$. Donde sup$\{a,b\}$ es el \textit{menor cota superior} o 
\textit{supremo}, e inf$\{a,b\}$ es la \textit{mayor cota inferior} o \textit{\'infimo}.

Un conjunto parcialmente ordenado $P$ es \textit{completo} si para cada subconjunto $A$ 
de $P$ existen sup $A$ e inf $A$. Todos los conjuntos parcialmente ordenados completos son 
ret\'iculos, y un ret\'iculo $L$ que es completo como conjunto parcialmente ordenado, es un 
\textit{ret\'iculo completo}.

Por ejemplo, si $L$ es un conjunto de proporciones, $\vee$ y $\wedge$ denotan 
la disjunci\'on y la conjunci\'on, respectivamente (i.e. ``or'' y ``and''). Entonces las 
identidades \textit{L1} a \textit{L4} son propiedades conocidasde la l\'ogica proposicional. 

The non-negative integers, ordered by divisibility. The least element of this lattice is the number 1, since it divides any other number. Perhaps surprisingly, the greatest element is 0, because it can be divided by any other number. The supremum of finite sets is given by the least common multiple and the infimum by the greatest common divisor. For infinite sets, the supremum will always be 0 while the infimum can well be greater than 1. For example, the set of all even numbers has 2 as the greatest common divisor. If 0 is removed from this structure it remains a lattice but ceases to be complete.

Un ejemplo de \textit{ret\'iculo completo} son los n\'umeros naturales, ordenados por divisibilidad. 
El supremo esta dado por el m\'inimo com\'un multiplo y el \'infimo por el m\'aximo com\'un divisor. Para conjuntos 
infinitos el supremo e \'infimo son el $0$ y $1$, respectivamente. 

Si se remueve el $0$ del conjunto anteriormente descrito, ya no es un \textit{ret\'iculo completo} puesto 
que no tiene supremo para todos los subconjuntos. Aunque sigue siendo un \textit{ret\'iculo}.

Para m\'as informaci\'on se refiere al lector a~\cite{burris1981course}.



