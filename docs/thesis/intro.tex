% !TEX root = main.tex

\chapter{Introducci\'on}
\label{cha.intro}

Programaci\'on concurrente por restricciones (del ingl\'es, \textit{Concurrent Constraint Programming}, abreviado $\CCP$)~\cite{DBLP:conf/popl/SaraswatR90,semantic-ccp,cp-book} es un modelo de concurrencia que combina el sistema operacional tradicional del c\'alculo de procesos con una idea declarativa basada en la l\'ogica. La cuesti\'on fundamental en $\CCP$ es la especificaci\'on de sistemas por medio de restricciones que representan informaci\'on parcial sobre algunas variables. El modelo $\CCP$ es param\'etrico en el sistema de restricciones especificando la estructura e interdependencias de la informaci\'on parcial que los procesos pueden solicitar y publicar en un banco compartido. En el modelo  $\CCP$ los procesos pueden ser considerados transmisores y el estado del sistema esta determinado por el banco de informaci\'on (i.e. una restricci\'on) que se refina monot\'onicamente por procesos que agregan nueva informaci\'on.

La noci\'on de sistema de restricciones se ha fortalecido recientemente con la inclusi\'on de funciones espaciales al lenguaje de restricciones, dando como resultado un sistema espacial por restricciones (del ingl\'es, \textit{Spatial Constraint System}, abreviado SCS)~\cite{knight:hal-00761116}. Las funciones espaciales pueden considerarse como operadores topol\'ogicos y de clausura que permiten la especificaci\'on de informaci\'on espacial y epist\'emica como parte de la estructura de restricciones. La programaci\'on espacial concurrente por restricciones (del ingl\'es, \textit{Spatial Concurrent Constraint Programming}, abreviado $\SCCP$) es un modelo de concurrencia resultado de parametrizar $\CCP$ con un sistema espacial por restricciones~\cite{knight:hal-00761116}. El modelo $\SCCP$ es ideal para modelar y razonar acerca de agentes que consultan y comparten informaci\'on en la presencia de jerarqu\'ias espaciales. Algunos ejemplos de gesti\'on de acceso a informaci\'on por jerarquias incluyen c\'irculos de amigos y publicaci\'on de mensajes en redes sociales, carpetas compartidas en la nube, y compartimentaci\'on de la ejecuci\'on de procesos en el sistema operativo. De estos campos surgen cuestiones como lo es el dise\~no de pol\'iticas para predecir y prevenir problemas de privacidad, por ejemplo.

Este documento presenta una sem\'antica en l\'ogica de reescritura que agrupa la sem\'antica operacional estructurada de $\SCCP$~\cite{knight:hal-00761116}. La sem\'antica en l\'ogica de reescritura de $\SCCP$ es una teor\'ia ejecutable en l\'ogica de reescritura~\cite{Meseguer199273}. La especificaci\'on en l\'ogica de reescritura puede ser ejecutada en Maude~\cite{maude-book}, un lenguaje e implementaci\'on de l\'ogica de reescritura de alto desempe\~no. La sem\'antica en l\'ogica de reescritura de $\SCCP$ hace parte de un esfuerzo para brindar t\'ecnicas deductivas y algor\'itmicas para razonar acerca de las propiedades espaciales y epist\'emicas de los sistemas concurrentes. Espec\'ificamente, la sem\'antica ejecutable en l\'ogica de reescritura presentada en este documento puede ser utilizada para ser ejecutada y verificar algoritmicamente propiedades de alcanzabilidad de sistemas de agentes dentro de una estructura espacial jer\'arquica y con informaci\'on localizada.

Finalmente, para poder poner los conceptos espaciales y epist\'emicos al alcance de los programadores, este documento introduce un prototipo de lenguaje de programaci\'on con una sintaxis simple que soporta sistemas $\SCCP$, tal como consultas y modificaci\'on de bancos de informaci\'on indexando por el espacio de un agente. Los programas escritos en este lenguaje pueden ser introducidos en una herramienta que se comunica con el sistema de Maude y puede ser ejecutada en la sem\'antica en l\'ogica de reescritura de $\SCCP$.

El documento est\'a organizado de la siguiente manera:
\begin{enumerate}
\item El Capitulo~\ref{p.prelim} incluye algunos conceptos requeridos para la lectura del documento.
\item El Cap\'itulo~\ref{chapter.sccp} presenta la descripci\'on del modelo $\SCCP$ y su sem\'antica operacional.
\item El Cap\'itulo~\ref{chapter.rew} presenta la especificaci\'on formal donde se modelan los agentes, procesos y sus transiciones. 
\item El Cap\'itulo~\ref{chapter.lang} presenta el lenguaje de programaci\'on propuesto basado en el modelo $\SCCP$.
\item El Cap\'itulo~\ref{chapter.envir} presenta el ambiente desarrollado para el lenguaje de programaci\'on.
\item El Anexo~\ref{a.appendix} incluye los m\'odulos de sistema y funcionales requeridos para la especificaci\'on del modelo $\SCCP$.
\end{enumerate}
