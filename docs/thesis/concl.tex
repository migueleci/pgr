% !TEX root = main.tex

\chapter{Conclusi\'on}
\label{c.conclu}

La l\'ogica de reescritura es una l\'ogica para especificar sistemas concurrentes la cual, al ser combinada con Maude y sus herramientas, facilita no solo la especificaci\'on formal de sistemas sino tambi\'en la verificaci\'on mec\'anica de las propiedades que debe satisfacer el sistema.

En este documento se presenta una sem\'antica en l\'ogica de reescritura para $\SCCP$ la cual es ejecutable en Maude. La sem\'antica ejecutable en l\'ogica de reescritura para $\SCCP$ descrita en este documento busca brindarle al programador un ambiente para an\'alizar sistemas distribuidos. 

El m\'odulo de reescritura SMT soluciona los problemas relacionados con la ejecuci\'on simb\'olica y la combinaci\'on entre reescritura de t\'erminos con la soluci\'on de problemas de reestricciones.

Como trabajo futuro se puede adaptar la sem\'antica en l\'ogica de reescritura para variaciones del modelo $\SCCP$, en donde se considere la existencia de mentiras transmitidas intencionalmente por agentes en el sistema. Extender la herramienta para brindarle al programador una traza de la evoluci\'on del sistema, con la que pueda observar paso a paso la ejecuci\'on de procesos y modificaci\'on de los bancos de informaci\'on. Es importante extender la herramienta incluyendo operaciones como extrusi\'on. Finalmente, se sugiere agregar m\'as elementos al lenguaje de programaci\'on, en los que se incluya la declaraci\'on de procesos y especificaci\'on de agentes a priori. 