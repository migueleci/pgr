\newtheorem{theorem}{Theorem}[chapter]
\newtheorem{lemma}[theorem]{Lemma}

\theoremstyle{definition}
\newtheorem{definition}[theorem]{Definici\'on}
%\newtheorem{definition}[section]{Definici\'on}
\newtheorem{example}[theorem]{Example}
\newtheorem{xca}[theorem]{Exercise}

\theoremstyle{remark}
\newtheorem{remark}[theorem]{Remark}

\numberwithin{section}{chapter}
\numberwithin{equation}{chapter}
\numberwithin{figure}{chapter}

%
\newcommand{\tbf}[1]{\textbf{#1}}
\newcommand{\nterm}[1]{\ensuremath{\left\langle{{#1}}\right\rangle}}
\newcommand{\textterm}[1]{\cde{`#1\textrm'}}
\newcommand{\term}[1]{{\pmb{#1}}}
\newcommand{\bor}{\phantom{::=}|\ }

% calligraphic 
\newcommand{\acal}{\mathcal{A}}
\newcommand{\bcal}{\mathcal{B}}
\newcommand{\ccal}{\mathcal{C}}
\newcommand{\dcal}{\mathcal{D}}
\newcommand{\ecal}{\mathcal{E}}
\newcommand{\fcal}{\mathcal{F}}
\newcommand{\gcal}{\mathcal{G}}
\newcommand{\hcal}{\mathcal{H}}
\newcommand{\ical}{\mathcal{I}}
\newcommand{\jcal}{\mathcal{J}}
\newcommand{\kcal}{\mathcal{K}}
\newcommand{\lcal}{\mathcal{L}}
\newcommand{\mcal}{\mathcal{M}}
\newcommand{\ncal}{\mathcal{N}}
\newcommand{\ocal}{\mathcal{O}}
\newcommand{\pcal}{\mathcal{P}}
\newcommand{\qcal}{\mathcal{Q}}
\newcommand{\rcal}{\mathcal{R}}
\newcommand{\scal}{\mathcal{S}}
\newcommand{\tcal}{\mathcal{T}}
\newcommand{\ucal}{\mathcal{U}}
\newcommand{\vcal}{\mathcal{V}}
\newcommand{\wcal}{\mathcal{W}}
\newcommand{\xcal}{\mathcal{X}}
\newcommand{\ycal}{\mathcal{Y}}
\newcommand{\zcal}{\mathcal{Z}}

%Sort
\newcommand{\states}{\mathfrak{s}}

%aux
\newcommand{\imps}{\Rightarrow}

%Tool
\newcommand{\tool}[1]{{\sf #1}}
\newcommand{\inva}{\tool{InvA}}

% auxiliary notation and functions for equations and rules
\newcommand{\EQ}{=}
\newcommand{\IFS}{\mathbf{if}}
\newcommand{\REW}{\rightarrow}
\newcommand{\eq}[2]{#1 \EQ #2}
\newcommand{\ceq}[3]{#1 \EQ #2 \; \IFS\; #3}
\newcommand{\rl}[2]{#1 \REW #2}
\newcommand{\crl}[3]{#1 \REW #2 \; \IFS\; #3}
\newcommand{\lbl}[1]{[\mathit{#1}]}
\newcommand{\cond}{\gamma}
\newcommand{\cuad}[1]{[#1]}
\newcommand{\tipo}[1]{[#1]_{E}}
\newcommand{\rel}[2]{\cuad{#1} \in T_{\Sigma / E, {#2}}}
\newcommand{\rels}[1]{\tipo{#1} \in T_{\Sigma / E, \states}}
\newcommand{\f}[1]{\rightarrow_{#1}}
\newcommand{\fa}[2]{\rightarrow_{#1}^{#2}}
\newcommand{\fl}[2]{\tipo{#1} \f{\rcal} \tipo{#2}}

% code
\newcommand{\id}[1]{\cde{#1}}
\newcommand{\cde}[1]{\textnormal{\texttt{#1}}}
\newcommand{\rlname}[1]{\cde{[#1]}}
\newcommand{\opname}[1]{\cde{#1}}
\lstdefinelanguage{Maude}{%
   keywords={
    , mod, fmod, endm, endfm
    , pr , protecting 
    , ex , extending 
    , inc, including
    , sort, sorts, subsort, subsorts
    , var, vars
    , op, ops
    , eq, ceq
    , rl, crl
    , if
    , search
    , red, reduce
    }
}
\lstnewenvironment{maude}
{\lstset{language=Maude ,
    keywordstyle=\color{blue}
  , basicstyle=\ttfamily\singlespacing\small
  , commentstyle={}
  , columns=flexible
  , numbers=none
  , showstringspaces=false
  , keepspaces=true
  , aboveskip=-3pt
  } 
}
{}

\DefineVerbatimEnvironment{maude2}{Verbatim}{fontsize=\scriptsize}

% code
\lstdefinelanguage{sccp}{%
   keywords={
    , begin, end
    , var, Int, Bool
    }
}
\lstnewenvironment{sccp}
{\lstset{language=SCCP ,
    keywordstyle=\color{blue}
  , basicstyle=\ttfamily\singlespacing\small
  , commentstyle={}
  , columns=flexible
  , numbers=none
  , showstringspaces=false
  , keepspaces=true
  , aboveskip=-3pt
  } 
}
{}

\DefineVerbatimEnvironment{sccp2}{Verbatim}{fontsize=\scriptsize}

%%%%%%%

\usepackage{proof} 
\usepackage{myproof}
\usepackage{mathtools}
\usepackage{bm}

\newcommand{\CS}{\textnormal{CS}}
\newcommand{\SCS}{\textnormal{SCS}}
\newcommand{\CCP}{\textnormal{CCP}}
\newcommand{\SCCP }{\textnormal{SCCP}}
\newcommand{\ECCP}{\textnormal{ECCP}}

\newcommand{\conf}[2]{\langle #1; #2\rangle}
\newcommand{\redi}{\longrightarrow }

\newcommand{\cleq}[2]{#1 \sqsubseteq #2}
\newcommand{\K}[2]{\left[ #2 \right] _{#1}}	
\DeclarePairedDelimiter{\sfunc}{\bm[}{\bm]}

\newcommand{\tellp}[1]{\tell(#1)}
\newcommand{\tell}{{\bf tell}}
\newcommand{\askp}[2]{\ask \  #1 \  \rightarrow \ #2}
\newcommand{\ask}{{\bf ask}}
\newcommand{\Stop}{{\bf 0}}

\newcommand{\rTell}{\rm R_{TELL}}
\newcommand{\rAsk}{\rm R_{ASK}}
\newcommand{\rPar}{\rm R_{PAR}}
\newcommand{\rSp}{\rm R_{SP}}

\newcommand{\defsymbol}{\stackrel{\textup{\texttt{def}}}  {=}}	% Definition symbol
